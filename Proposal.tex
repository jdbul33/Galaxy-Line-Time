\documentclass[12pt]{article}
\usepackage[margin=1in]{geometry}
\usepackage[utf8]{inputenc}
\usepackage{graphicx}
\usepackage{pdfpages}
\usepackage{hyperref}
\usepackage{setspace}
\usepackage[title]{appendix}
\graphicspath{ {./Figures/} }
\hypersetup{
	colorlinks=true,
	linkcolor=blue,
	filecolor=magenta,      
	urlcolor=cyan,
}
\urlstyle{same}
\usepackage[font={small,it}]{caption}
\usepackage{fancyvrb}
\title{Time Series Analysis \& Forecasting of Samsung Galaxy Search Trends}
\author{John D. Bulger, Craig Garzella\\
	Analytics \& Modeling
	\\Valparaiso University, Valparaiso, IN\thanks{``We have neither given or received, nor have we tolerated others' use of unauthorized aid."}}
\date{Presentation: April 30, 2019}

\begin{document}
	\maketitle
	
	\section{Introduction}
	
	Samsung, a global conglomerate, offers an enormous product line.  These include computers, appliances, wearables, and mobile devices, among many others.  Their Galaxy line, premium mobile phones, often have the highest consumer levels impact and attention.  The annual releases of the Galaxy S and Galaxy Note devices over the past several years garner a large amount of speculatory articles, leaks, and reviews.  Upon release, many service providers and retailers offer pre-order and purchase deals and bundles with these devices.
	
	\par
	
	
	Google Trends is a free service that offers an unparalleled look into Google Search activity.  Any search term can be viewed over a custom time frame, filtered by demographic region.  Famously, this service gave rise to the Google Flu Trends, where the start of flu season can be identified earlier by analyzing Googlers' symptom searches.  Similarly, this service allows for search trends to be shown not just for exact terms, but on general topics.  Results are shown as a time series, with points shown as relative search levels.
	
	\par
	
	The goal of this analysis is to develop an explanatory time series model for the search impact of the Samsung Galaxy product line.  Seasonality, as well as a general trend, are expected to be present in such data.  By creating an appropriate ARIMA model, it will then be possible to predict search impact for the next product release cycle in 2020.
	
	\section{Data}
	
	The data for this analysis consist of 88 data points spanning 2012-2019 for Google searches for the ``Samsung Galaxy" product line.  Each point is a monthly search index level ranging from 2014 to 2019.  The data is freely available at the Google Trends site by following \href{https://trends.google.com/trends/explore?q=\%2Fm\%2F0hnbsn3\&geo=US}{this link}.
	

\newpage
\begin{thebibliography}{3}
	
	
\end{thebibliography}

\newpage
\begin{appendices}	
	\section{Modeling Data Subset}
	
\newpage

	\section{Forecasting Data Subset}

\end{appendices}




\end{document}